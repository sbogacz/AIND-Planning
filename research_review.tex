\documentclass{article}
\usepackage[margin=1.00in]{geometry}
\usepackage{listings}
\usepackage{color}
\usepackage{amsmath}

\title{A Brief Introduction to Distributed Planning}
\author{Steven Bogacz}
\date{September 2017}
\begin{document}
\maketitle
\begin{section}{Introduction}
	The issues inherent in multi-agent systems are appealing to me in general, as I spend most
	of my time at work working on a distributed multi-agent text classifying system. In many
	ways, we can interpret neural networks as being a way of programatticaly dealing with multi-agent
	systems. This led me to look for overviews of multi-agent approaches in order to solve planning
	problems. The challenges inherent in multi-agent systems stem from the need to support the 
	interleaving of actions in order to guarantee completeness, as was pointed out by Allen Brown (Sussman, 1975).
	This challenge is only exacerbated when we the agents are also decentralized, with no main finalizing agent
	to aggregate the results. Furthermore, a given agent in a distributed multi-agent system 
	isn't only concerned with the question of how to interleave its actions with those of its 
	companions, but it must also deal with the possibiliy that each agent might view a different subset of
	the problem space, if only partial observability can be guaranteed.\cite{bonisoli} 
\end{section}
\begin{section}{Developments}

\end{section}
\begin{thebibliography}{9}

	\bibitem{bonisoli}
	Andrea Bonisoli,
	\textit{Distributed and Multi-Agent Planning: Challenges and Open Issues},
	Universita degli Studi di Brescia,
	Brescia, Italy
  2013.
  \bibitem{nissim}
  Raz Nissim et al.,
  \textit{A General, Fully Distributed Multi-Agent Planning Algorithm},
  Ben-Gurion University,
  Israel
  2012.

\end{thebibliography}
\end{document}
