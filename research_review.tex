\documentclass{article}
\usepackage[margin=1.00in]{geometry}
\usepackage{listings}
\usepackage{color}
\usepackage{amsmath}

\title{A Brief Introduction to Distributed Planning}
\author{Steven Bogacz}
\date{September 2017}
\begin{document}
\maketitle
\begin{section}{Introduction}
	The issues inherent in multi-agent systems are appealing to me in general, as I spend most
	of my time at work working on a distributed multi-agent text classifying system. In many
	ways, we can interpret neural networks as being a way of programatticaly dealing with multi-agent
	systems. This led me to look for overviews of multi-agent approaches in order to solve planning
	problems. The challenges inherent in multi-agent systems stem from the need to support the 
	interleaving of actions in order to guarantee completeness, as was pointed out by Allen Brown (Sussman, 1975).
	This challenge is only exacerbated when we the agents are also decentralized, with no main finalizing agent
	to aggregate the results. Furthermore, a given agent in a distributed multi-agent system 
	isn't only concerned with the question of how to interleave its actions with those of its 
	companions, but it must also deal with the possibiliy that each agent might view a different subset of
	the problem space, if only partial observability can be guaranteed.\cite{bonisoli} 
\end{section}
\begin{section}{Developments}
	One of the first works to start branching heavily from previous approaches is to combine distributed multi-agent
	approaches with CSP problem solving approaches. In Nissim et al.'s work\cite{nissim} they expand on Brafman \& Domshlak's 
	work of applying CSPs to multi-agent planning problems\cite{brafman}. They note that despite the theoretical completeness
	of Brafman \& Domshlak's approach (as most CSP solution mechanisms often resort to some search based approach, they can
	guarantee completeness and optimality via-BFS), the requirement of their DisCSP creates nonbinary constraints on otherwise
	independent agents.\cite{nissim}\par
	They use two mechanisms by which to separate variable definiton. The first is to separate the agent-specific notion of 
	\textbf{Action Variables}, which correspond to the actions internally available to the agent, \textbf{Time Variables}, which provide 
	ordered sequences of times when an action may be performed, and \textbf{Requirement Variables}, which provide a mechanism by
	which external agents can satisfy the requirements for a given action. The second distinction is to extend the role of each agent in their 
	internal planning. Rather than limiting themselves to their internally determined cootdination points (as per\cite{brafman}), they utilize
	all private and public actions, in an "ignore preconditions"-like approach to distributed multi-agent planning. This in turn determines
	their internal ordering of the aforementioned variables, and is refined in an action-first order as other agens report conflicts in the
	planning.\par
	Another novel idea in the realm of distributed multi-agent planning is the identification of heuristics which are additive across agents
	yet retain their admissibility, i.e. they do not overestimate the cost of attaining a goal state.\cite{stolba} Specifically, these works 
	focus on multi-agent cost-partitioning, wherein the cost of actions must be non-negative, and less than, or equal to, the actual cost of
	a given action. This allows some information transfer between agents, while retaining relatively strong measures of privacy (although the 
	article's authors do indicate some vagueness around weak and strong variable privacy).\cite{stolba} This idea appears to still require 
	robust testing, but preliminary findings would appear to show promise in its applicability to various scenarious (although it appears
	to be very sensitive to the given heuristic.
	
\end{section}
\begin{thebibliography}{9}

	\bibitem{bonisoli}
	Andrea Bonisoli,
	\textit{Distributed and Multi-Agent Planning: Challenges and Open Issues},
	Universita degli Studi di Brescia,
	Brescia, Italy
  2013.
  @ http://ceur-ws.org/Vol-1126/paper6.pdf
  \bibitem{nissim}
  Raz Nissim et al.,
  \textit{A General, Fully Distributed Multi-Agent Planning Algorithm},
  Ben-Gurion University,
  Israel
  2010.
  @https://www.cs.bgu.ac.il/~raznis/AAMAS2010\_0192\_dd86c6585.pdf
  \bibitem{brafman}
  Ronen I. Brafman \& Carmel Domshlak
  \textit{From One to Many: Planning for Loosely Coupled Multi-Agent Systems}
  Bun-Gurion University,
  Israel
  2008
  @ https://www.aaai.org/Papers/ICAPS/2008/ICAPS08-004.pdf
  \bibitem{stolba}
  Michal Stolba \& Antonin Komenda
  Department of Computer Science, Faculty of Electrical Engineering,
  Czech Technical University in Prague, Czech Republic
  @ http://icaps16.icaps-conference.org/proceedings/dmap16.pdf\#page=92
\end{thebibliography}
\end{document}
